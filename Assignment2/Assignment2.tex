
  
\documentclass[journal,12pt,twocolumn]{IEEEtran}

\usepackage{setspace}
\usepackage{gensymb}
\singlespacing
\usepackage[cmex10]{amsmath}

\usepackage{amsthm}
\usepackage{commath}
\usepackage{mathrsfs}
\usepackage{txfonts}
\usepackage{stfloats}
\usepackage{bm}
\usepackage{cite}
\usepackage{cases}
\usepackage{subfig}

\usepackage{longtable}
\usepackage{multirow}

\usepackage{enumitem}
\usepackage{mathtools}
\usepackage{steinmetz}
\usepackage{tikz}
\usepackage{circuitikz}
\usepackage{verbatim}
\usepackage{tfrupee}
\usepackage[breaklinks=true]{hyperref}
\usepackage{graphicx}
\usepackage{tkz-euclide}

\usetikzlibrary{calc,math}
\usepackage{listings}
    \usepackage{color}                                            %%
    \usepackage{array}                                            %%
    \usepackage{longtable}                                        %%
    \usepackage{calc}                                             %%
    \usepackage{multirow}                                         %%
    \usepackage{hhline}                                           %%
    \usepackage{ifthen}                                           %%
    \usepackage{lscape}     
\usepackage{multicol}
\usepackage{chngcntr}

\DeclareMathOperator*{\Res}{Res}

\renewcommand\thesection{\arabic{section}}
\renewcommand\thesubsection{\thesection.\arabic{subsection}}
\renewcommand\thesubsubsection{\thesubsection.\arabic{subsubsection}}

\renewcommand\thesectiondis{\arabic{section}}
\renewcommand\thesubsectiondis{\thesectiondis.\arabic{subsection}}
\renewcommand\thesubsubsectiondis{\thesubsectiondis.\arabic{subsubsection}}


\hyphenation{op-tical net-works semi-conduc-tor}
\def\inputGnumericTable{}                                 %%

\lstset{
%language=C,
frame=single, 
breaklines=true,
columns=fullflexible
}
\begin{document}

\newcommand{\BEQA}{\begin{eqnarray}}
\newcommand{\EEQA}{\end{eqnarray}}
\newcommand{\define}{\stackrel{\triangle}{=}}
\bibliographystyle{IEEEtran}
\raggedbottom
\setlength{\parindent}{0pt}
\providecommand{\mbf}{\mathbf}
\providecommand{\pr}[1]{\ensuremath{\Pr\left(#1\right)}}
\providecommand{\qfunc}[1]{\ensuremath{Q\left(#1\right)}}
\providecommand{\sbrak}[1]{\ensuremath{{}\left[#1\right]}}
\providecommand{\lsbrak}[1]{\ensuremath{{}\left[#1\right.}}
\providecommand{\rsbrak}[1]{\ensuremath{{}\left.#1\right]}}
\providecommand{\brak}[1]{\ensuremath{\left(#1\right)}}
\providecommand{\lbrak}[1]{\ensuremath{\left(#1\right.}}
\providecommand{\rbrak}[1]{\ensuremath{\left.#1\right)}}
\providecommand{\cbrak}[1]{\ensuremath{\left\{#1\right\}}}
\providecommand{\lcbrak}[1]{\ensuremath{\left\{#1\right.}}
\providecommand{\rcbrak}[1]{\ensuremath{\left.#1\right\}}}
\theoremstyle{remark}
\newtheorem{rem}{Remark}
\newcommand{\sgn}{\mathop{\mathrm{sgn}}}
\providecommand{\abs}[1]{\vert#1\vert}
\providecommand{\res}[1]{\Res\displaylimits_{#1}} 
\providecommand{\norm}[1]{\lVert#1\rVert}
%\providecommand{\norm}[1]{\lVert#1\rVert}
\providecommand{\mtx}[1]{\mathbf{#1}}
\providecommand{\mean}[1]{E[ #1 ]}
\providecommand{\fourier}{\overset{\mathcal{F}}{ \rightleftharpoons}}
%\providecommand{\hilbert}{\overset{\mathcal{H}}{ \rightleftharpoons}}
\providecommand{\system}{\overset{\mathcal{H}}{ \longleftrightarrow}}
	%\newcommand{\solution}[2]{\textbf{Solution:}{#1}}
\newcommand{\solution}{\noindent \textbf{Solution: }}
\newcommand{\cosec}{\,\text{cosec}\,}
\providecommand{\dec}[2]{\ensuremath{\overset{#1}{\underset{#2}{\gtrless}}}}
\newcommand{\myvec}[1]{\ensuremath{\begin{pmatrix}#1\end{pmatrix}}}
\newcommand{\mydet}[1]{\ensuremath{\begin{vmatrix}#1\end{vmatrix}}}
\numberwithin{equation}{subsection}
\makeatletter
\@addtoreset{figure}{problem}
\makeatother
\let\StandardTheFigure\thefigure
\let\vec\mathbf
\renewcommand{\thefigure}{\theproblem}
\def\putbox#1#2#3{\makebox[0in][l]{\makebox[#1][l]{}\raisebox{\baselineskip}[0in][0in]{\raisebox{#2}[0in][0in]{#3}}}}
     \def\rightbox#1{\makebox[0in][r]{#1}}
     \def\centbox#1{\makebox[0in]{#1}}
     \def\topbox#1{\raisebox{-\baselineskip}[0in][0in]{#1}}
     \def\midbox#1{\raisebox{-0.5\baselineskip}[0in][0in]{#1}}
\vspace{3cm}
\title{Assignment 2}
\author{Adarsh Sai - AI20BTECH11001}
\maketitle
\newpage
\bigskip
\renewcommand{\thefigure}{\theenumi}
\renewcommand{\thetable}{\theenumi}
Download all python codes from 
\begin{lstlisting}
https://github.com/Adarsh541/EE3900/blob/main/Assignment2/codes/Assignment2.py
\end{lstlisting}

%
Download latex-tikz codes from 
%
\begin{lstlisting}
https://github.com/Adarsh541/EE3900/blob/main/Assignment2/Assignment2.tex
\end{lstlisting}
\section{Problem(Matrices Q2.7)}
If, $\Vec{A}=\myvec{1&2&-3\\5&0&2\\1&-1&1}$, $\Vec{B}=\myvec{3&-1&2\\4&2&5\\2&0&3}$ and $\Vec{C}=\myvec{4&1&2\\0&3&2\\1&-2&3}$, then compute $\brak{\Vec{A+B}}$ and $\brak{\Vec{B-C}}$. Also, verify that $\Vec{A}+\brak{\Vec{B-C}}=\brak{\Vec{A+B}}-\Vec{C}$
\section{Solution}
\begin{align}
   \Vec{A+B}&=\myvec{1&2&-3\\5&0&2\\1&-1&1}+\myvec{3&-1&2\\4&2&5\\2&0&3}\\
   &=\myvec{1+3&2+(-1)&-3+2\\5+4&0+2&2+5\\1+2&-1+0&1+3}\\
   &=\myvec{4&1&-1\\9&2&7\\3&-1&4}
\end{align}
\begin{align}
    \Vec{B-C}&=\myvec{3&-1&2\\4&2&5\\2&0&3}-\myvec{4&1&2\\0&3&2\\1&-2&3}\\
    &=\myvec{3-4&-1-1&2-2\\4-0&2-3&5-2\\2-1&0-(-2)&3-3}\\
    &=\myvec{-1&-2&0\\4&-1&3\\1&2&0}
\end{align}
\begin{align}
    \Vec{A}+\brak{\Vec{B-C}}&=\myvec{1&2&-3\\5&0&2\\1&-1&1}+\myvec{-1&-2&0\\4&-1&3\\1&2&0}\\
    &=\myvec{1+(-1)&2+(-2)&-3+0\\5+4&0+(-1)&2+3\\1+1&-1+2&1+0}\\
    &=\myvec{0&0&-3\\9&-1&5\\2&1&1}\label{2.0.9}
\end{align}
\begin{align}
    \brak{\Vec{A+B}}-\Vec{C}&=\myvec{4&1&-1\\9&2&7\\3&-1&4}-\myvec{4&1&2\\0&3&2\\1&-2&3}\\
    &=\myvec{4-4&1-1&-1-2\\9-0&2-3&7-2\\3-1&-1-(-2)&4-3}\\
    &=\myvec{0&0&-3\\9&-1&5\\2&1&1}\label{2.0.12}
\end{align}
$\eqref{2.0.9}$ is same as $\eqref{2.0.12}$
\end{document}

