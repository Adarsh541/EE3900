
  
\documentclass[journal,12pt,twocolumn]{IEEEtran}

\usepackage{setspace}
\usepackage{gensymb}
\singlespacing
\usepackage[cmex10]{amsmath}

\usepackage{amsthm}

\usepackage{mathrsfs}
\usepackage{txfonts}
\usepackage{stfloats}
\usepackage{bm}
\usepackage{cite}
\usepackage{cases}
\usepackage{subfig}

\usepackage{longtable}
\usepackage{multirow}

\usepackage{enumitem}
\usepackage{mathtools}
\usepackage{steinmetz}
\usepackage{tikz}
\usepackage{circuitikz}
\usepackage{verbatim}
\usepackage{tfrupee}
\usepackage[breaklinks=true]{hyperref}
\usepackage{graphicx}
\usepackage{tkz-euclide}

\usetikzlibrary{calc,math}
\usepackage{listings}
    \usepackage{color}                                            %%
    \usepackage{array}                                            %%
    \usepackage{longtable}                                        %%
    \usepackage{calc}                                             %%
    \usepackage{multirow}                                         %%
    \usepackage{hhline}                                           %%
    \usepackage{ifthen}                                           %%
    \usepackage{lscape}     
\usepackage{multicol}
\usepackage{chngcntr}

\DeclareMathOperator*{\Res}{Res}
\newtheorem{theorem}{Theorem}[section]
\newtheorem{corollary}{Corollary}[theorem]
\newtheorem{lemma}[theorem]{Lemma}
\newtheorem{definition}{Definition}[section]
\renewcommand\thesection{\arabic{section}}
\renewcommand\thesubsection{\thesection.\arabic{subsection}}
\renewcommand\thesubsubsection{\thesubsection.\arabic{subsubsection}}

\renewcommand\thesectiondis{\arabic{section}}
\renewcommand\thesubsectiondis{\thesectiondis.\arabic{subsection}}
\renewcommand\thesubsubsectiondis{\thesubsectiondis.\arabic{subsubsection}}


\hyphenation{op-tical net-works semi-conduc-tor}
\def\inputGnumericTable{}                                 %%

\lstset{
%language=C,
frame=single, 
breaklines=true,
columns=fullflexible
}
\begin{document}

\newcommand{\BEQA}{\begin{eqnarray}}
\newcommand{\EEQA}{\end{eqnarray}}
\newcommand{\define}{\stackrel{\triangle}{=}}
\bibliographystyle{IEEEtran}
\raggedbottom
\setlength{\parindent}{0pt}
\providecommand{\mbf}{\mathbf}
\providecommand{\pr}[1]{\ensuremath{\Pr\left(#1\right)}}
\providecommand{\qfunc}[1]{\ensuremath{Q\left(#1\right)}}
\providecommand{\sbrak}[1]{\ensuremath{{}\left[#1\right]}}
\providecommand{\lsbrak}[1]{\ensuremath{{}\left[#1\right.}}
\providecommand{\rsbrak}[1]{\ensuremath{{}\left.#1\right]}}
\providecommand{\brak}[1]{\ensuremath{\left(#1\right)}}
\providecommand{\lbrak}[1]{\ensuremath{\left(#1\right.}}
\providecommand{\rbrak}[1]{\ensuremath{\left.#1\right)}}
\providecommand{\cbrak}[1]{\ensuremath{\left\{#1\right\}}}
\providecommand{\lcbrak}[1]{\ensuremath{\left\{#1\right.}}
\providecommand{\rcbrak}[1]{\ensuremath{\left.#1\right\}}}
\theoremstyle{remark}
\newtheorem{rem}{Remark}
\newcommand{\sgn}{\mathop{\mathrm{sgn}}}
\providecommand{\abs}[1]{\vert#1\vert}
\providecommand{\res}[1]{\Res\displaylimits_{#1}} 
\providecommand{\norm}[1]{\lVert#1\rVert}
%\providecommand{\norm}[1]{\lVert#1\rVert}
\providecommand{\mtx}[1]{\mathbf{#1}}
\providecommand{\mean}[1]{E[ #1 ]}
\providecommand{\fourier}{\overset{\mathcal{F}}{ \rightleftharpoons}}
%\providecommand{\hilbert}{\overset{\mathcal{H}}{ \rightleftharpoons}}
\providecommand{\system}{\overset{\mathcal{H}}{ \longleftrightarrow}}
	%\newcommand{\solution}[2]{\textbf{Solution:}{#1}}
\newcommand{\solution}{\noindent \textbf{Solution: }}
\newcommand{\cosec}{\,\text{cosec}\,}
\providecommand{\dec}[2]{\ensuremath{\overset{#1}{\underset{#2}{\gtrless}}}}
\newcommand{\myvec}[1]{\ensuremath{\begin{pmatrix}#1\end{pmatrix}}}
\newcommand{\mydet}[1]{\ensuremath{\begin{vmatrix}#1\end{vmatrix}}}
\numberwithin{equation}{subsection}
\makeatletter
\@addtoreset{figure}{problem}
\makeatother
\let\StandardTheFigure\thefigure
\let\vec\mathbf
\renewcommand{\thefigure}{\theproblem}
\def\putbox#1#2#3{\makebox[0in][l]{\makebox[#1][l]{}\raisebox{\baselineskip}[0in][0in]{\raisebox{#2}[0in][0in]{#3}}}}
     \def\rightbox#1{\makebox[0in][r]{#1}}
     \def\centbox#1{\makebox[0in]{#1}}
     \def\topbox#1{\raisebox{-\baselineskip}[0in][0in]{#1}}
     \def\midbox#1{\raisebox{-0.5\baselineskip}[0in][0in]{#1}}
\vspace{3cm}
\title{GATE Assignment 4}
\author{Adarsh Sai - AI20BTECH11001}
\maketitle
\newpage
\bigskip
\renewcommand{\thefigure}{\theenumi}
\renewcommand{\thetable}{\theenumi}
Download all python codes from 
\begin{lstlisting}
https://github.com/Adarsh541/EE3900/blob/main/Gate4/codes/Gate4.py
\end{lstlisting}

%
Download latex-tikz codes from 
%
\begin{lstlisting}
https://github.com/Adarsh541/EE3900/blob/main/Gate4/Gate4.tex
\end{lstlisting}
\section{Problem(GATE 2001 EC Q.2.19)}
The Nyquist sampling interval, for the signal
\begin{align}
    x\brak{t} = Sinc(700t)+Sinc(500t)
\end{align}
is.
\begin{enumerate}
    \item $\frac{1}{350}$sec\\
    \item $\frac{\pi}{350}$sec\\
    \item $\frac{1}{700}$sec\\
    \item $\frac{\pi}{175}$sec
\end{enumerate}
\section{Solution}
\begin{lemma}{Fourier transform of Sinc function}
\begin{align}
    Sinc\brak{at} \fourier \frac{1}{|a|}rect\brak{\frac{f}{a}}\label{ft}
\end{align}
\end{lemma}
Since Fourier Transform is linear, and using $\eqref{ft}$
\begin{align}
    X\brak{f}&=\frac{1}{700}rect\brak{\frac{f}{700}}+\frac{1}{500}rect\brak{\frac{f}{500}}\\
    X\brak{f}&=0  \text{   for } f>350Hz
\end{align}
\begin{align}
    \text{Nyquist rate} &= 2 \times \text{max frequency}\\
    &= 2\times 350 Hz\\
    \text{Nyquist sampling interval} &= \frac{1}{\text{Nyquist rate}}\\
    &= \frac{1}{700}sec
\end{align}
\begin{center}
\begin{tabular}{ |m{2cm}|m{4.5cm}| } 
 \hline
 Sampling rate(Hz) & Observations \\
 \hline
 $fs=1k$& The signal is perfectly reconstructed. It is made up of frequencies $-350<f<350$. \\
 \hline
 $fs=2k$ & The signal is perfectly reconstructed. It is made up of frequencies $-350<f<350$. \\
 \hline
 $fs=500$ & $150<f<350$ are missing   \\ 
 \hline
 $fs=600$ & $250<f<350$ are missing  \\ 
 \hline
\end{tabular}
\end{center}
\begin{figure}[!h]
 \centering
 \includegraphics[width=\columnwidth]{sinc.png}
 \caption{Plot of $x\brak{t}$ sampled at 1kHz.} 
\end{figure}
\begin{figure}[!h]
 \centering
 \includegraphics[width=\columnwidth]{gnyq.png}
 \caption{DFT of $x\brak{t}$ sampled at 1k and 2k} 
\end{figure}
\begin{figure}[!h]
 \centering
 \includegraphics[width=\columnwidth]{lnyq.png}
 \caption{DFT of $x\brak{t}$ sampled at 600 and 500Hz.} 
\end{figure}
\end{document}

