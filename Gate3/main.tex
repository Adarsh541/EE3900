
  
\documentclass[journal,12pt,twocolumn]{IEEEtran}

\usepackage{setspace}
\usepackage{gensymb}
\singlespacing
\usepackage[cmex10]{amsmath}

\usepackage{amsthm}
\usepackage{commath}
\usepackage{mathrsfs}
\usepackage{txfonts}
\usepackage{stfloats}
\usepackage{bm}
\usepackage{cite}
\usepackage{cases}
\usepackage{subfig}

\usepackage{longtable}
\usepackage{multirow}

\usepackage{enumitem}
\usepackage{mathtools}
\usepackage{steinmetz}
\usepackage{tikz}
\usepackage{circuitikz}
\usepackage{verbatim}
\usepackage{tfrupee}
\usepackage[breaklinks=true]{hyperref}
\usepackage{graphicx}
\usepackage{tkz-euclide}
\usepackage{array}
\usetikzlibrary{calc,math}
\usepackage{listings}
    \usepackage{color}                                            %%
    \usepackage{array}                                            %%
    \usepackage{longtable}                                        %%
    \usepackage{calc}                                             %%
    \usepackage{multirow}                                         %%
    \usepackage{hhline}                                           %%
    \usepackage{ifthen}                                           %%
    \usepackage{lscape}     
\usepackage{multicol}
\usepackage{chngcntr}

\DeclareMathOperator*{\Res}{Res}

\renewcommand\thesection{\arabic{section}}
\renewcommand\thesubsection{\thesection.\arabic{subsection}}
\renewcommand\thesubsubsection{\thesubsection.\arabic{subsubsection}}

\renewcommand\thesectiondis{\arabic{section}}
\renewcommand\thesubsectiondis{\thesectiondis.\arabic{subsection}}
\renewcommand\thesubsubsectiondis{\thesubsectiondis.\arabic{subsubsection}}
\newtheorem{theorem}{Theorem}[section]
\newtheorem{corollary}{Corollary}[theorem]
\newtheorem{lemma}[theorem]{Lemma}
\newtheorem{definition}{Definition}[section]

\hyphenation{op-tical net-works semi-conduc-tor}
\def\inputGnumericTable{}                                 %%

\lstset{
%language=C,
frame=single, 
breaklines=true,
columns=fullflexible
}
\begin{document}

\newcommand{\BEQA}{\begin{eqnarray}}
\newcommand{\EEQA}{\end{eqnarray}}
\newcommand{\define}{\stackrel{\triangle}{=}}
\bibliographystyle{IEEEtran}
\raggedbottom
\setlength{\parindent}{0pt}
\providecommand{\mbf}{\mathbf}
\providecommand{\pr}[1]{\ensuremath{\Pr\left(#1\right)}}
\providecommand{\qfunc}[1]{\ensuremath{Q\left(#1\right)}}
\providecommand{\sbrak}[1]{\ensuremath{{}\left[#1\right]}}
\providecommand{\lsbrak}[1]{\ensuremath{{}\left[#1\right.}}
\providecommand{\rsbrak}[1]{\ensuremath{{}\left.#1\right]}}
\providecommand{\brak}[1]{\ensuremath{\left(#1\right)}}
\providecommand{\lbrak}[1]{\ensuremath{\left(#1\right.}}
\providecommand{\rbrak}[1]{\ensuremath{\left.#1\right)}}
\providecommand{\cbrak}[1]{\ensuremath{\left\{#1\right\}}}
\providecommand{\lcbrak}[1]{\ensuremath{\left\{#1\right.}}
\providecommand{\rcbrak}[1]{\ensuremath{\left.#1\right\}}}
\theoremstyle{remark}
\newtheorem{rem}{Remark}
\newcommand{\sgn}{\mathop{\mathrm{sgn}}}
\providecommand{\abs}[1]{\vert#1\vert}
\providecommand{\res}[1]{\Res\displaylimits_{#1}} 
\providecommand{\norm}[1]{\lVert#1\rVert}
%\providecommand{\norm}[1]{\lVert#1\rVert}
\providecommand{\mtx}[1]{\mathbf{#1}}
\providecommand{\mean}[1]{E[ #1 ]}
\providecommand{\fourier}{\overset{\mathcal{F}}{ \rightleftharpoons}}
%\providecommand{\hilbert}{\overset{\mathcal{H}}{ \rightleftharpoons}}
\providecommand{\system}{\overset{\mathcal{H}}{ \longleftrightarrow}}
	%\newcommand{\solution}[2]{\textbf{Solution:}{#1}}
\newcommand{\solution}{\noindent \textbf{Solution: }}
\newcommand{\cosec}{\,\text{cosec}\,}
\providecommand{\dec}[2]{\ensuremath{\overset{#1}{\underset{#2}{\gtrless}}}}
\newcommand{\myvec}[1]{\ensuremath{\begin{pmatrix}#1\end{pmatrix}}}
\newcommand{\mydet}[1]{\ensuremath{\begin{vmatrix}#1\end{vmatrix}}}
\numberwithin{equation}{subsection}
\makeatletter
\@addtoreset{figure}{problem}
\makeatother
\let\StandardTheFigure\thefigure
\let\vec\mathbf
\renewcommand{\thefigure}{\theproblem}
\def\putbox#1#2#3{\makebox[0in][l]{\makebox[#1][l]{}\raisebox{\baselineskip}[0in][0in]{\raisebox{#2}[0in][0in]{#3}}}}
     \def\rightbox#1{\makebox[0in][r]{#1}}
     \def\centbox#1{\makebox[0in]{#1}}
     \def\topbox#1{\raisebox{-\baselineskip}[0in][0in]{#1}}
     \def\midbox#1{\raisebox{-0.5\baselineskip}[0in][0in]{#1}}
\vspace{3cm}
\title{Gate Assignment 3}
\author{Adarsh Sai - AI20BTECH11001}
\maketitle
\newpage
\bigskip
\renewcommand{\thefigure}{\theenumi}
\renewcommand{\thetable}{\theenumi}
Download all python codes from 
\begin{lstlisting}
https://github.com/Adarsh541/EE3900/blob/main/Gate3/codes/Gate3.py
\end{lstlisting}

%
Download latex-tikz codes from 
%
\begin{lstlisting}
https://github.com/Adarsh541/EE3900/blob/main/Gate2/Gate3.tex
\end{lstlisting}
\section{Problem(Gate EC 2005 Q.23)}
The power in the signal $s\brak{t}=8\cos\brak{20\pi t-\frac{\pi}{2}}+4\sin\brak{15\pi t}$ is
\begin{enumerate}
    \item 40
    \item 41
    \item 42
    \item 82
\end{enumerate}
\section{Solution}
\begin{lemma}\label{sum}
If two sinusoids have sufficiently different frequencies, the power of the sum is the sum of powers.
\end{lemma}
\begin{theorem}\label{pow}
The power of a sinusoidal signal $x\brak{t}=A\sin\brak{2\pi ft}$ is $\frac{A^2}{2}$.
\end{theorem}
\begin{proof}
\begin{align}
    P &= \lim_{T\to\infty}\frac{1}{2T}\int_{-T}^{T}\brak{x\brak{t}}^2dt\\
    \int_{-T}^{T}\brak{x\brak{t}}^2dt&=\int_{-T}^{T}\brak{\sin\brak{2\pi ft}}^2dt\\
    &=\int_{-T}^{T}\frac{A^2}{2}\brak{1-\cos\brak{4\pi ft}}dt\\
    &=\sbrak{\frac{A^2}{2}\brak{t-\frac{\sin\brak{4\pi ft}}{4\pi ft}}}_{-T}^{T}\\
    &=A^2T\\
    \implies P&=\lim_{T\to\infty}\frac{1}{2T}\brak{A^2T}\\
    &=\frac{A^2}{2}
\end{align}
\end{proof}
\begin{align}
    s\brak{t}&=8\cos\brak{20\pi t-\frac{\pi}{2}}+4\sin\brak{15\pi t}\\
    &=8\sin\brak{20\pi t}+4\sin\brak{15\pi t}
\end{align}
From lemma-$\ref{sum}$ and Theorem-$\ref{pow}$, power in $s\brak{t}$ is 
\begin{align}
    P &= \frac{8^2}{2}+\frac{4^2}{2}\\
    &=40
\end{align}
\begin{figure}[!h]
 \centering
 \includegraphics[width=\columnwidth]{signal.png}
 \caption{Plot of $s\brak{t}$.}
\end{figure}
\end{document}

